\documentclass[a4paper]{article}

\usepackage[utf8]{inputenc}
\usepackage[polish]{babel}
\usepackage{polski}
\usepackage{ dsfont }
\usepackage[top=2cm, bottom=2cm, left=2cm, right=2cm]{geometry}

\begin{document}

\setlength{\parindent}{0pt}

\textbf{Zadanie 16.} Czy dla każdego języka języka regularnego istnieje deterministyczne on-line wyrażenie regularne, które go definiuje?
\vspace{1cm}

Wyrażenie regularne $\psi$ \textbf{opisuje} język $L_{\phi}$ jeśli $L_{\psi} = L_{\phi}$.

Język jest \textbf{zdegenerowany}, jeśli jest pusty lub składa się jedynie ze słowa pustego.
\vspace{1cm}

\textbf{Rozwiązanie.} Nie. Weźmy język opisywany przez wyrażenie $(000)^* + (00000)^*$. Wyrażenie regularne nad alfabetem $\Sigma = \{ 0 \}$ może być postaci:

$$ \o,\hspace{5mm} \epsilon,\hspace{5mm} 0,\hspace{5mm} \phi^*,\hspace{5mm} \phi + \psi,\hspace{5mm} \phi \circ \psi$$

dla $\phi, \psi$ będących wyrażeniami regularnymi. Dla każdej z powyższych postaci pokażemy, że takie wyrażenie regularne albo nie opisuje języka $L_{(000)^* + (00000)^*}$ albo nie jest deterministyczne on-line.

\begin{enumerate}
\item $\o$, $\epsilon$ i $0$. Języki opisywane przez te wyrażenia są skończone, a język $L_{(000)^* + (00000)^*}$ jest nieskończony.
\item $\phi^*$. Żeby to wyrażenie opisywało $L_{(000)^* + (00000)^*}$, do języka $L_{\phi}$ musi należeć zarówno $000$ jak i $00000$. Wtedy $00000000$ ($3+5$ zer) należy do $L_{\phi^*}$ a nie należy do $L_{(000)^* + (00000)^*}$. To wyrażenie nie opisuje jezyka $L_{(000)^* + (00000)^*}$.
\item $\phi + \psi$. Rozpatrzmy przypadki:
  \begin{itemize}
    \item $\phi$ lub $\psi$ jest zdegenerowany. Niech będzie to $\psi$, przypadek dla $\phi$ jest symetryczny. Skoro $\psi$ jest zdegenerowany to $\phi$ opisuje $L_{(000)^* + (00000)^*}$. Powtarzamy więc nasze rozumowanie dla wyrażenia regularnego $\phi$. Wyrażenia regularne są skończonymi drzewami oraz $\phi$ jest mniejsze niż $\psi + \phi$ (np. w porządku bycia poddrzewem, ale też w porządku, w którym porównujemy liczności wierzchołków), więc nasze rozumowanie nie będzie powtarzać się w nieskończoność.
    \item $\phi$ i $\psi$ zawierają przynajmniej po jednym niepustym słowie. Weźmy $w_1 \in L_{\phi}, w_2 \in L_{\psi}$. Słowo składające się z jednego zera jest zarówno prefiksem $w_1$, jak i prefiksem $w_2$. W pierwszym przypadku to zero mapuje się na pewien symbol z $\phi$, a w drugim przypadku mapuje się na pewien symbol z $\psi$. Wyrażenie nie jest deterministyczne on-line.
  \end{itemize}
\item $\phi \circ \psi$. Rozpatrzmy przypadki:
  \begin{itemize}
  \item $\phi$ lub $\psi$ jest zdegenerowany. Jeśli degeneracja polega na tym, że język $\phi$ (symetrycznie $\psi$) jest pusty to konkatenacja języka pustego z dowolnym językiem daje język pusty a $L_{(000)^* + (00000)^*}$ jest niepusty. Jeśli degeneracja polega na tym, że język $\phi$ (symetrycznie $\psi$) składa się jedynie z $\epsilon$ to powtarzamy rozumowanie dla $\psi$.
    \item $\phi$ i $\psi$ opisują języki skończone. Wtedy ich konkatenacja nie daje języka nieskończonego, jakim jest $L_{(000)^* + (00000)^*}$.
    \item $\phi$ opisuje język o tylko jednym elemencie $\epsilon$, a $\psi$ opisuje język nieskończony. Wtedy powtarzamy rozumowanie dla $\psi$.
    \item $\phi$ opisuje język o tylko jednym elemencie, różnym od $\epsilon$, a $\psi$ opisuje język nieskończony. Wtedy ich konkatenacja nigdy nie da słowa pustego, które należy do $L_{(000)^* + (00000)^*}$
    \item $\phi$ opisuje język składający się z dwóch słów, $\epsilon$ i słowa niepustego $0^k$, a $\psi$ opisuje język nieskończony. Żeby ta konkatenacja opisywała wybrany język to $L_{\psi} \cup L_{0^k \circ \psi} = L_{(000)^* + (00000)^*}$.

\textbf{Lemat.} W $L_{\psi}$ znajduje się zarówno słowo $000$ jak i słowo $00000$. 

\textbf{Dowód.} Jeśli w $L_{\psi}$ nie ma słowa $000$ to zachodzą przypadki:
\begin{itemize}
\item $\epsilon \in L_{\psi}, k = 3$. Wtedy aby wyrazić $0^{10}$ to samo $0^{10}$ jest w $L_{\psi}$ lub $0^7$ jest w języku i wyrażamy $0^{10}$ przez dopisanie trzech zer do $0^7$. W pierwszym przypadku $0^{13}$ też musiałoby być w $L_{(000)^* + (00000)^*}$, a w drugim przypadku już $0^7$ musiałoby być w $L_{(000)^* + (00000)^*}$.
\item $0 \in L_{\psi}, k = 2$. Analogicznie do przypadku powyżej, chcemy otrzymać $0^6$, a ani $0^4$ ani $0^8$ nie ma w $L_{(000)^* + (00000)^*}$.
\item $00 \in L_{\psi}, k = 1$. Analogicznie do przypadku powyżej, chcemy otrzymać $0^{15}$, a ani $0^{14}$ ani $0^{16}$ nie ma w $L_{(000)^* + (00000)^*}$.
\end{itemize}

Jeśli w $L_{\psi}$ nie ma słowa $00000$ to zachodzą przypadki:
\begin{itemize}
\item $\epsilon \in L_{\psi}, k = 5$. Wtedy chcemy otrzymać $0^6$, a ani $0^1$ ani $0^{11}$ nie ma w $L_{(000)^* + (00000)^*}$.
\item $0 \in L_{\psi}, k = 4$. Wtedy chcemy otrzymać $0^{12}$, a ani $0^8$ ani $0^{16}$ nie ma w $L_{(000)^* + (00000)^*}$.
\item $00 \in L_{\psi}, k = 3$. Dowód jak dla braku $000$.
\item $000 \in L_{\psi}, k = 2$. Dowód jak dla braku $000$.
\item $0000 \in L_{\psi}, k = 1$. Dowód jak dla braku $000$.
\end{itemize}
\vspace{1cm}

Weźmy $w_1 \in L_{\psi}, w_2 = 0^k \circ w_1$. Rozpatrzmy podprzypadki:
      \begin{itemize}
        \item $k$ nie jest podzielne przez $3$.
        \item $k$ nie jest podzielne przez $5$.
        \item $k$ jest podzielne przez $15$. Wtedy wyrażenie nie jest deterministyczne (więc w szczególności nie jest deterministyczne on-line), ponieważ wszystkie zera słowa $0^{15}$ mogą mapować się na symbole $\phi$ lub symbole z $\psi$.
      \end{itemize}
    \item $\phi$ opisuje język o co najmniej dwóch elementach różnych od $\epsilon$, a $\psi$ opisuje język niezdegenerowany. Weźmy dwa takie słowa $krotsze, dluzsze \in L_{\phi}$ takie, że $|krotsze| < |dluzsze|$. Weźmy dowolne niepuste słowo $w \in L_{psi}$. Rozpatrzmy słowo $prefiks = krotsze \circ 0$ i podkreślmy ostatnie jego zero. Słowo $prefiks$ jest prefiksem zarówno słowa $krotsze \circ w$ jak i słowa $dluzsze \circ w$. W pierwszym przypadku podkreślone zero mapuje się na pewien symbol z $\psi$ a w drugim przypadku mapuje się na pewien symbol z $\phi$.
  \end{itemize}
\end{enumerate}

\end{document}