\documentclass[12pt,a4paper]{article}

\usepackage[utf8]{inputenc}
\usepackage[polish]{babel}
\usepackage{polski}
\usepackage{graphics}
\usepackage{amsmath}
\usepackage{mathrsfs}
\usepackage{hyperref}

\makeatletter
\renewcommand\@seccntformat[1]{\csname the#1\endcsname.\quad}
\renewcommand\numberline[1]{#1.\hskip0.7em}

\linespread{1.5}

\newcommand{\linia}{\rule{\linewidth}{0.4mm}}

\renewcommand{\maketitle}{\begin{titlepage}
    \vspace*{1cm}
    \begin{center}
      Instutut Informatyki Uniwersytetu Wrocławskiego\\
      Przemoc seksualna w konfliktach zbrojnych \\
      Prowadzący: dr Arkadiusz Domagała \\
      \vspace{3cm}
      \normalsize \@author \par
      \vspace{0.8cm}
      \noindent
      \LARGE \textsc{\@title}\\
      \vspace{1cm}
      \normalsize
    \end{center}
    \vspace{0.5cm}
    \begin{flushright}
      \vspace{5cm}
    \end{flushright}
    \vspace*{\stretch{6}}
    \begin{center}
      \@date
    \end{center}
  \end{titlepage}%
}

\makeatother
\title{Analiza przypadku Stevena Greena}
\author{Maciej Pacut, nr indeksu: 221082}
\date{Wrocław 2012}


\begin{document}
\maketitle
\newpage

\section{Tekst}

Przypadek miał miejsce podczas interwencji pokojowej w Iraku
(2003-2010). Czwórka amerykańskich żołnierzy: Steven Green, James
Barker, Paul Cortez, Jesse Spielman dnia 12 marca 2006 roku wtargnęła
do irackiego domu w wiosce Yusufiyah niedaleko miesta Al-Mahmudiyah.
Według materiału dowodowego żołnierze zgwłcili a następnie zabili
14-letnią Abeer Qasim Hamza oraz zabili jej matkę, Fakhriyah Taha
Muhsin (34 lata), ojca Qasim Hamza Raheem (45 lat) oraz siostrę Hadeel
Qasim Hamza (6 lat).


Zgodnie z zeznaniami sąsiadów Aber spędzała większość dnia w domu -
rodzice nie zezwalali na uczęszczanie Aber do szkoły z powodu sytuacji
zagrażającej bezpieczeństwu. Brat Aber, Mohammed, który w trakcie wtargnięcia
znajdował się w szkole zeznał, że żołnierze często przeszukiwali dom.
Podczas pewnego przeszukania szeregowy Green przesunął wskazujący
palec po policzku Aber; to zachowanie przeraziło dziewczynkę. Sąsiedzi
ostrzegali ojca Aber o nadmiernym zainteresowaniu żołnierzy jego
córką, co zostało zignorowane przez ojca, gdyż według niego Aber była
jeszcze zbyt młodą dziewczynką by sytuacja była dla niej
niebezpieczna. Według zeznan bliskich Aber jej matka widywała
żołnierzy wpatrujących się w jej córkę oraz pokazujących jej (matce)
gest uniesionego kciuka, zapewniający, że wszystko jest w porządku.
Zainteresowanie żołnierzy córką spowodowało, że matka poleciła jej
spędzać noce w domu swojego wujka Ahmada Qassima. W trakcie śledztwa
żołnierze złożyli zeznania potwierdzające, że planowali przestępstwo z
wyprzedzeniem dni.


Dom, w którym popełniono przestępstwo był oddalony 200 metrów od
posterunku sił stabilizacyjnych.



\section{Plan yebood}

\begin{enumerate}
\item przedstawienie głównego 'bohatera'
\item jego zajęcia
\item czyny będące podstawą do oskarżenia
\item dokładny opis i przebieg procesu wraz z wynikiem
\item twoja opinia o czynie, oskarżonym, wyroku oraz jego ogólnym postępowaniu
\end{enumerate}

\section{Plan własny}

\begin{enumerate}
\item konteks historyczny, epokowy (jaka wojan, między kim a kim)
\item bohater - kawałek życiorysu do rozpoczęcia wojny
\item bohater - przebieg służby wojskowej
\item bohater - szczegóły zbrodni
\item jak zbrodnia wyszłą na jaw
\item przebieg procesu, wyrok
\item opinia Malwiny o czynie
\end{enumerate}

\section{Źródła}

\begin{enumerate}
\item ``Gwałt jest tańszy niż kule'', Maciej Konarski (\url{http://trzeciswiat.blox.pl/2009/12/Gwalt-jest-tanszy-niz-kule.html})
\item ``BBC Documentary on Mahmudiya Massacre'',
  (\url{http://www.expose-the-war-profiteers.org/DOD/iraq_II_videos/mahmudiya_documentary.htm})
\item ``Troops 'took turns' to rape Iraqi'' (\url{http://news.bbc.co.uk/2/hi/middle_east/5253160.stm})
\item \url{freepaulcortez.org}
\item \url{dontfreepaulcortez.org}
\item ``100 lat dla żołnierza za zbrodnię w Iraku'' (\url{http://wyborcza.pl/1,86746,3943052.html})
\end{enumerate}

\end{document}