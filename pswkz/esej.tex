
\documentclass[12pt,a4paper]{article}

\usepackage[utf8]{inputenc}
\usepackage[polish]{babel}
\usepackage{polski}
\usepackage{graphics}
\usepackage{amsmath}
\usepackage{mathrsfs}
\usepackage{hyperref}
\usepackage{fancyhdr}

\makeatletter
\renewcommand\@seccntformat[1]{\csname the#1\endcsname.\quad}
\renewcommand\numberline[1]{#1.\hskip0.7em}

\linespread{1.5}

\pagestyle{fancyplain}
 
\fancyhf{}
\lhead{\fancyplain{}{Maciej Pacut, 221082, Analiza przypadku Stevena Greena}}

\makeatother
\title{Analiza sprawy Stevena Greena}
\author{Maciej Pacut, nr indeksu: 221082}
\date{Wrocław 2012}


\begin{document}

\begin{center}
\LARGE{\textbf{Analiza sprawy Stevena Greena}}
\end{center}

Zbrodnia miała miejsce podczas interwencji pokojowej w Iraku
(2003-2010). Czwórka amerykańskich żołnierzy: Steven Green, James
Barker, Paul Cortez, Jesse Spielman dnia 12 marca 2006 roku wtargnęła
do irackiego domu w wiosce Yusufiyah niedaleko miesta Al-Mahmudiyah.
Według materiału dowodowego żołnierze zgwłcili a następnie zabili
14-letnią Abeer Qasim Hamza oraz zabili jej matkę, Fakhriyah Taha
Muhsin (34 lata), ojca Qasim Hamza Raheem (45 lat) oraz siostrę Hadeel
Qasim Hamza (6 lat).


Zgodnie z zeznaniami sąsiadów Abeer spędzała większość dnia w domu -
rodzice nie zezwalali na uczęszczanie Abeer do szkoły z powodu sytuacji
zagrażającej jej bezpieczeństwu. Brat Abeer, Mohammed, który w trakcie wtargnięcia
znajdował się w szkole, zeznał, że żołnierze często przeszukiwali dom.
Podczas pewnego przeszukania szeregowy Green przesunął wskazujący
palec po policzku Abeer; to zachowanie przeraziło dziewczynkę. Sąsiedzi
ostrzegali ojca Abeer o nadmiernym zainteresowaniu żołnierzy jego
córką, co zostało zignorowane przez ojca, gdyż według niego Abeer była
jeszcze zbyt młodą dziewczynką by sytuacja była dla niej
niebezpieczna. Według zeznan bliskich Abeer jej matka widywała
żołnierzy wpatrujących się w jej córkę oraz pokazujących jej (matce)
gest uniesionego kciuka, zapewniający, że wszystko jest w porządku.
Zainteresowanie żołnierzy córką spowodowało, że matka poleciła jej
spędzać noce w domu swojego wujka Ahmada Qassima. W trakcie śledztwa
żołnierze złożyli zeznania potwierdzające, że planowali przestępstwo z
wyprzedzeniem.


Dom, w którym popełniono przestępstwo był oddalony 200 metrów od
posterunku sił stabilizacyjnych. 12 marca 2006 roku sprawcy
opisywanego przestępstwa pełnili służbę na posterunku, pijąc alkohol i
dyskutując o planach gwałtu na Abeer. Z sześcioosobowego składu
posterunku obecnych było pięciu żołnirzy, z czego Brian Howard nie
brał bezpośredniego udziału we wtargnięciu - był jednak świadomy
planów żołnierzy i poszedł z nimi w pobliże domu by ostrzegać
towarzyszy przed niebezpieczeństwem. Żołnierze wtargnęli do domu i
umieścili Abeer w jednym pokoju a jej rodzinę w drugim. Green
zamordował matkę, ojca i siostrę Abeer a w tym czasie pozostali
żołnierze gwałcili Abeer. Green wrócił do pozostałych żołnierzy
mówiąc: ``Właśnie ich zabiłem, wszyscy nie żyją.``. Następnie zgwałcił
Abeer i zabił ją strzałem w głowę. W celu ukrycia swojej zbrodni
żołnierze usiłowali spalić ciała.


Po opuszczeniu domu przez żołnierzy sąsiedzi zauważyli obłoki dymu
wydobywające się z okien oraz martwe ciała w środku. Sąsiedzi udali
się do wuja Abeer, Abu Firas Janabiego, który przybiegł do domu Abeer.
Po przybyciu Abu Firas Janabi zgasił część płomieni w celu dostania
się do środka. Po ujrzeniu sceny Abu Firas Janabi udał się do
posterunku sił stablizacyjnych - innego, niż ten, przy którym
stacjonowali sprawcy - w celu zgłoszenia przestępstwa. Po około godzinie
iraccy żołnierze dostarli do domu Abeer. Towarzyszył im sprawca,
Steven Green. Po przybyciu na miejsce zbrodni Steven Green powiedział irackim
żołnierzom, że winnymi zbrodni są sunniccy powstańcy. Iraccy żołnierze
udzielili takiego wyjaśnienia wujowi Abeer.


16 czerwca 2006 roku brygada, do której należeli sprawcy została zaatakowana i
dwóch jej członków zginęło. 11 lipca 2006 roku organizacja sunnicka Mujahideen
Shura Council wypuściła materiał wideo pokazujący ciała zamordowanych żołnierzy
wraz z komentarzem, że atak był odpowiedzią na dyshonor, który zadał
ich siostrze żołnierz tej samej brygady. Sierżant Anthony Yribe, który
był członkiem tej samej brygady doniósł po ataku organizacji sunnickiej, że sprawcą zbrodni
jest Steven Green. Nie udało się ustalić, skąd organizacja sunnicka
dowiedziała się, kto był sprawcą zbrodni.


Do czasu złożenia przez sierżanta Anthonego Yribe zeznań oskarżających Steven Green został
wydalony z Armii Stanów Zjednoczonych z powodu ``antysocjalnych
zaburzeń osobowości''. 6 lipca 2006 roku na pierwszej rozprawie Green
został uznany za niewinnego. Skutkiem rozgłosu medialnego, jaki
towarzyszył sprawie, 31 sierpnia 2006 roku uznano apelację prokuratorską i
sprawa została ponownie otwarta. 4 września 2009 roku Green został
skazany na karę śmierci.


Moim zdaniem przestępstwo dokonane przez Stevena Greena to wyjątkowo odrażający przykład
zbrodni wojennej. To, czego dopuścił się amerykański żołnierz, wydaje
się być zbrodnią tak szczególną nie tylko dlatego, że na jej dokonanie
składały się dwa etapy: gwałt oraz morderstwo, ale i z tego powodu, że
poprzedzało ją skrupulatne planowanie, a co gorsza - ideologiczna
podbudowa. Podczas drugiej apelacji Green wyznał, że według niego
``Irakijczycy to nie ludzie''. Jego nieludzkie poglądy oraz
bestialskie czyny sprawiają, że według mnie zasługuje on na wymierzony
wymiar kary.

Źródła:
\begin{enumerate}
\item ``Gwałt jest tańszy niż kule'', Maciej Konarski (\url{http://trzeciswiat.blox.pl/2009/12/Gwalt-jest-tanszy-niz-kule.html})
\item ``BBC Documentary on Mahmudiya Massacre'',
  (\url{http://www.expose-the-war-profiteers.org/DOD/iraq_II_videos/mahmudiya_documentary.htm})
\item ``Troops 'took turns' to rape Iraqi'' (\url{http://news.bbc.co.uk/2/hi/middle_east/5253160.stm})
\item \url{freepaulcortez.org}
\item \url{dontfreepaulcortez.org}
\item ``100 lat dla żołnierza za zbrodnię w Iraku'' (\url{http://wyborcza.pl/1,86746,3943052.html})
\end{enumerate}

\end{document}