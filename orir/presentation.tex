\documentclass{beamer}
\usetheme{Antibes}
\usecolortheme{seagull}
\setbeamertemplate{footline}{\hfill\insertframenumber/\inserttotalframenumber\hspace*{.5cm}}
\setbeamertemplate{navigation symbols}{}

\usepackage[utf8]{inputenc}
\usepackage[T1]{fontenc}
\usepackage[polish]{babel}
\usepackage{hyperref}
\usepackage{rotating}
\usepackage{listings}

\title{Obliczenia równoległe i rozproszone}
\author{Maciej Pacut}
\date{Luty 2012}

\AtBeginSubsection[]
{
  \begin{frame}<beamer>
    \frametitle{Layout}
    \tableofcontents[currentsection,currentsubsection]
  \end{frame}
}

\begin{document}

\begin{frame}
  \titlepage
\end{frame}

\begin{frame}
\tableofcontents[allowframebreaks]
\end{frame}

\section{Wstęp}
\begin{frame}{Temat prezentacji}
  A. Karbowski, Programowanie równoległe i rozproszone

  Rozdziały 1,2,3
\end{frame}
\section{Rozdział 1}

\begin{frame}{Zalety obliczeń równoległych}
  \begin{enumerate}[<+->]
    \item przyspieszenie czasu przeprowadzania obliczeń
    \item zwiększenie niezawodności obliczeń
    \item stworzenie możliwości rozwiązania zadań zbyt dużych dla maszyny sekwencyjnej
    \item umożliwienie wykorzystania pełnej mocy obliczeniowej procesorów wielordzeniowych
  \end{enumerate}
\end{frame}

\begin{frame}{Podstawowe pojęcia}
  \begin{enumerate}
    \item Obliczenia współbieżne - następne zadanie rozpoczyna się zanim skończy się poprzednie (mogą być przeprowadzane na jednym procesorze z podziałem czasu)
    \item Obliczenia równoległe - obliczenia współbieżne na wielu procesorach
    \item Obliczenia rozproszone - obliczenia bez dzielonej pamięci (przekazywanie komunikatów)
  \end{enumerate}
\end{frame}

\begin{frame}{Przykłady problemów, których rozwiązanie jest możliwe tylko za pomocą algorytmów rozproszonych}
  \begin{enumerate}
    \item Prognoza pogody - dwa dni po
    \item Dane z teleskopów
  \end{enumerate}
\end{frame}

\section{Rozdział 2}

\begin{frame}{Prawo Amdahla}
  \begin{enumerate}
    \item Mówi o ograniczeniu górnym przyspieszenia wykonywania zadania przy przenoszeniu z procesora sekwencyjnego na wieloprocesory
    \item Wzorek, opis zmiennych i funkcji
  \end{enumerate}
\end{frame}
\section{Rozdział 3}
\end{document}
