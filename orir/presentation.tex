\documentclass{beamer}
\usetheme{Copenhagen}

\usepackage[utf8]{inputenc}
\usepackage[T1]{fontenc}
\usepackage[polish]{babel}
\usepackage{hyperref}
\usepackage{rotating}
\usepackage{listings}

\title{Obliczenia równoległe i rozproszone}
\author{Maciej Pacut}
\date{Luty 2012}

\begin{document}

\maketitle

\begin{frame}{Zalety obliczeń równoległych}
  \begin{enumerate}
    \item przyspieszenie czasu przeprowadzania obliczeń
    \item zwiększenie niezawodności obliczeń
    \item stworzenie możliwości rozwiązania zadań zbyt dużych dla maszyny sekwencyjnej
    \item umożliwienie wykorzystania pełnej mocy obliczeniowej procesorów wielordzeniowych
  \end{enumerate}
\end{frame}

\begin{frame}{Podstawowe pojęcia}
  \begin{enumerate}
    \item Obliczenia współbieżne - następne zadanie rozpoczyna się zanim skończy się poprzednie
    \item Obliczenia równoległe - obliczenia współbieżne na wielu procesorach
    \item Obliczenia rozproszone - obliczenia bez wspólnej pamięci
  \end{enumerate}
\end{frame}
\end{document}
