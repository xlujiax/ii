\documentclass{beamer}
\usetheme{Copenhagen}

\usepackage[utf8]{inputenc}
\usepackage[T1]{fontenc}
\usepackage[polish]{babel}
\usepackage{hyperref}
\usepackage{rotating}
\usepackage{listings}

\title{Obliczenia równoległe i rozproszone}
\author{Maciej Pacut}
\date{Luty 2012}

\begin{document}

\maketitle

\begin{frame}{Zalety obliczeń równoległych}
  \begin{enumerate}
    \item przyspieszenie czasu przeprowadzania obliczeń
    \item \pause zwiększenie niezawodności obliczeń
    \item \pause stworzenie możliwości rozwiązania zadań zbyt dużych dla maszyny sekwencyjnej
    \item \pause umożliwienie wykorzystania pełnej mocy obliczeniowej procesorów wielordzeniowych
  \end{enumerate}
\end{frame}

\begin{frame}{Podstawowe pojęcia}
  \begin{enumerate}
    \item Obliczenia współbieżne - następne zadanie rozpoczyna się zanim skończy się poprzednie (mogą być przeprowadzane na jednym procesorze z podziałem czasu)
    \item Obliczenia równoległe - obliczenia współbieżne na wielu procesorach
    \item Obliczenia rozproszone - obliczenia bez dzielonej pamięci (przekazywanie komunikatów)
  \end{enumerate}
\end{frame}

\begin{frame}{Przykłady problemów, których rozwiązanie jest możliwe tylko za pomocą algorytmów rozproszonych}
  \begin{enumerate}
    \item Prognoza pogody - dwa dni po
  \end{enumerate}
\end{frame}
\end{document}
